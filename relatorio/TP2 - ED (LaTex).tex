\documentclass[11pt]{article}
\usepackage[utf8]{inputenc}
\usepackage[brazilian]{babel}
\usepackage{hyperref}
\usepackage{lipsum}
\usepackage{geometry}
\usepackage[skip=5pt plus1pt, indent=20pt]{parskip}
\usepackage{graphicx}

\usepackage{amsthm,amssymb,amsmath}


\newtheorem*{theorem}{Theorem}

\newcommand{\NN}{\mathbb{N}}
\newcommand{\ZZ}{\mathbb{Z}}
\newcommand{\RR}{\mathbb{R}}
\newcommand{\QQ}{\mathbb{Q}}
\newcommand{\CC}{\mathbb{C}}

 \title{\textbf{Trabalho Prático 2}}
\author{Sarah Menks Sperber}
\date{Novembro 2024}
 
\usepackage{fancyhdr}
\fancypagestyle{plain}{%  the preset of fancyhdr 
    \fancyhf{} % clear all header and footer fields
    \fancyfoot[L]{\thedate}
    \fancyhead[L]{DCC205 - Estrutura de Dados}
    \fancyhead[R]{Universidade Federal de Minas Gerais}
}
\makeatletter
\def\@maketitle{%
  \newpage
  \null
  \vskip 1em%
  \begin{center}%
  \let \footnote \thanks
    {\LARGE \@title \par}%
    \vskip 1em%
    %{\large \@date}%
  \end{center}%
  \par
  \vskip 1em}
\makeatother

\usepackage{lipsum}  
\usepackage{cmbright}

%%%%%%%%%%%%%%%%%%%%%%%%%%%%%%%%%%%%%%%%%%%%%%%%%%%%%%%%%%%%%%%%%%%%%%%%%%%%%%

\begin{document}

\maketitle

\noindent\begin{tabular}{@{}ll}
    Estudante: Sarah Menks Sperber\\
    Matrícula: 2023001824\\
    Email: sarahmenksperber@ufmg.com\\
\end{tabular}

\section*{Introdução:}

    \par Este trabalho consiste em desenvolver e implementar um sistema de escalonamento hospitalar para o Hospital Zambs (HZ). O objetivo principal é modelar o fluxo de atendimento do hospital, que abrange triagem, atendimento médico e execução de procedimentos, utilizando a técnica de simulação de eventos discretos. Essa abordagem permite monitorar o tempo de permanência dos pacientes no hospital, bem como avaliar o uso de recursos em diferentes cenários.\\
    Assim, no decorrer deste relatório, vamos fazer uma análise profunda dos algoritmos usados, discutindo pontos como metodologias, escolhas na implementação e complexidade assintótica.


%%%%%%%%%%%%%%%%%%%%%%%%%%%%%%%%%%%%%%%%%%%%%%%%%%%%%%%%%%%%%%%%%%%%%%%%%%%%%%
\section*{Métodos:}
\par Para a construção do sistema hospitalar, foram usados quatro TADs: Patiant.hpp, Procedure.hpp, Queue.hpp e Scheduler.hpp, onde se encontram as declarações das funções do código. Nesta seção, vamos discorrer um pouco sobre o papel de cada TAD e os respetivos algoritmos usados.

\subsection*{1. Patiant.hpp}
\par O TAD Patiant.hpp é encarregado por criar a struct Patiant, que armazena as informações do paciente, que são lidas no arquivo. Esse TAD não possui muitas funções, apenas métodos para impressão das informações do paciente e para a consulta dos dados (funções get).

\subsection*{2. Procedure.hpp}
\par Esse é o TAD responsável por armazenar as informações dos procedimentos, como tempo de execução e quantidade de unidades, e por implementar as funções que realizam o processamento dos procedimentos.

\subsection*{3. Queue.hpp}
\par Este TAD define a estrutura de dados Lista, projetada para armazenar e manipular os pacientes de forma sequencial. Sua implementação de deu atraves de uma lista encadeada, e aqui se encontram os métodos para a inserção e remoção dos elementos.

\subsection*{4. Scheduler.hpp}
\par O arquivo define a interface para o Escalonador, responsável por gerenciar a fila de prioridade utilizada na simulação do sistema hospitalar. Ele organiza os eventos pendentes de execução através de um min heap. Este arquivo inclui a definição das estruturas, métodos e atributos necessários para manipular eventos, como a inserção de novos eventos, remoção do próximo evento e consulta ao próximo evento na fila de prioridade

%%%%%%%%%%%%%%%%%%%%%%%%%%%%%%%%%%%%%%%%%%%%%%%%%%%%%%%%%%%%%%%%%%%%%%%%%%%%%%

\section*{Análise de Complexidade}
    \par Para facilitar a análise, vamos entender a complexidade assintótica das funções de acordo com o TAD que elas se encontram.

    \par\textbf{Patiant.hpp:} O TAD Patiants possui apenas 5 funções  - ConfigDate, PrintStatistics, GetPatiantTime e GetProcedureTime - que retornam ou formatam as informações de um paciente. Como seu tempo de execução não muda de acordo com o tamanho da entrada, essas funções são O(1).

    \par\textbf{Procedure.hpp:}
    \begin{itemize}
        \item Inicialize(): Atualiza os tempos de duração e de unidades - O(1) - e atualiza o estado de todas as unidades. Logo, a função é O(k), onde k é o numero de unidades.
        \item GetTime(): Retorna o tempo de duração do procedimento, O(1).
        \item FindEmptyUnit(): Percorre o vetor de unidades e vê qual unidade já está disponivel para o atendimento. Complexidade é O(k), onde k é o número de unidades.
        \item PerformProcedure(): A função apenas muda os valores do própio procedimento, mas como ele utiliza a função FindEmptyUnit(), sua complexidade é O(k)
    \end{itemize}

    \par\textbf{Patiant.hpp:}
    \begin{itemize}
        \item ConfigDate(): Apenas muda a formatação da data. Por isso, sua complexidade é O(n)
        \item PrintStatistics(): Imprime as estatísticas de um paciente, logo, é O(1)
        \item GetPatiantTime(): Apenas retorna o tempo total do paciente. O(1)
        \item GetProcedureTime(): Retorna o tempo do procedimento de acordo com o estado do paciente. O(1)
    \end{itemize}

    \par\textbf{Queue.hpp:}
    \begin{itemize}
        \item Enqueue(): Insere um novo elemento no final da fila. O(1)
        \item Remove(): Remove o primeiro elemento da lista. Logo, é O(1)
        \item isEmpty(): Verifica se o tamanho da lista é igual a 0, portanto, a função é O(1)
        Patiant* First(): Retorna o primeiro elemento da lista (no caso, o elemento apontado pelo head). Sua complexidade é O(1)
    \end{itemize}

    \par\textbf{Scheduler.hpp:}
    \begin{itemize}
        \item CreateEvent(): Cria um novo evento e, depois, o insere no escalonador. Para manter as propriedades do min heap, o método chama a função LowHeapfy, que é O(log n)
        \item InsertEvent(): Insere um evento no escalonador. Como ele também chama a função LowHeapfy, sua complexidade assintótica é O(log n)
        \item RemoveNext(): Remove o nó da árvore e chama a função HighHeapfy para manter as propriedades do min heap. Assim, sua complexidade é O(log n)
        \item GetParent(), GetLeftSucessor() e GetRightSucessor(): Como o min heap foi implementado com um vetor, essas funções são O(1)
        \item isEmpty(): erifica se o tamanho da lista é igual a 0, portanto, a função é O(1)
        \item GetNextTime(): Retorna a data do evento que está no nó da árvore. Assim, a função é O(1)
    \end{itemize}

%%%%%%%%%%%%%%%%%%%%%%%%%%%%%%%%%%%%%%%%%%%%%%%%%%%%%%%%%%%%%%%%%%%%%%%%%%%%%%%

\section*{Estratégias de Robustez:}
\par Para garantir um código seguro e sem falhas, algumas estratégias de segurança foram implementadas,
tais como:

\par\textbf{1. Uso de try catch:} Para tratar as excessões que podem ser lançadas, foram implementados blocos com instruções try, throw e catch.

\par\textbf{2. Verificação de arquivo válido:} A função main possui um teste de checagem do arquivo. Se o usuário não passar a entrada desejada por parâmetro (um arquivo válido), o código é
encerrado e uma mensagem de texto é exibida.

\par\textbf{3. Validações de entrada:} Além das funções mencionadas acima, o código também valida o arquivo de entrada e  confere os parâmetros passados entre as funções.


%%%%%%%%%%%%%%%%%%%%%%%%%%%%%%%%%%%%%%%%%%%%%%%%%%%%%%%%%%%%%%%%%%%%%%%%%%%%%%%

\section*{Análise Experimental:}

\par A análise do uso temporal do código visa avaliar o desempenho do programa em relação ao tempo de execução das suas operações. Para isso, foram realizados testes experimentais, medindo o tempo gasto por funções e processos principais, considerando diferentes tamanhos de entrada.
\par As Figuras 1 e 2 mostram o desempenho temporal do código, em nanossegundos.

\begin{figure}
    \centering
    \includegraphics[width=0.55\linewidth]{ProcessProcedure.png}
    \caption{Tempo gasto para processar 100 pacientes}
    \label{fig:enter-label}
\end{figure}

\begin{figure}
    \centering
    \includegraphics[width=0.55\linewidth]{Grafico codigo geral.png}
    \caption{Tempo geral para processar 100 pacientes}
    \label{fig:enter-label}
\end{figure}


%%%%%%%%%%%%%%%%%%%%%%%%%%%%%%%%%%%%%%%%%%%%%%%%%%%%%%%%%%%%%%%%%%%%%%%%%%%%%%%

\section*{Conclusão}    
    \par Por meio deste trabalho, foram consolidados conhecimentos teóricos sobre simulações de eventos discretos e sobre o impacto da implementação de soluções computacionais no gerenciamento de sistemas. O aprendizado obtido reforça a habilidade de lidar com problemas reais, criando sistemas robustos e eficientes. Além disso, foi possível praticar os conceitos relacionados a árvores binárias (min heap) e filas de prioridade, assuntos discutidos em aula. 
    
    \par Durante a implementação da solução para o problema, houveram alguns desafios que dificultaram a elaboração do trabalho, como o cálculo preciso dos tempos em fila.
    
    \par Toda a análise revelou a importância de decisões eficientes no escalonamento e alocação de recursos para minimizar tempos de espera e melhorar a utilização de recursos. Por isso, mesmo com os erros cometidos, a tarefa executada foi de grande valor para o meu aprendizado enquanto aluna.


%%%%%%%%%%%%%%%%%%%%%%%%%%%%%%%%%%%%%%%%%%%%%%%%%%%%%%%%%%%%%%%%%%%%%%%%%%%%%%%

\section*{Bibliografia:}

    \par LACERDA, Anísio; MEIRA JR., Wagner. Slides virtuais da disciplina de estruturas de dados. Belo Horizonte: Universidade Federal de Minas Gerais, 2024.
    
    \par VALGRIND. Valgrind User Manual. Disponível em:
https://valgrind.org/docs/manual/ms-manual.html. Acesso em: 2 dez. 2024

%%%%%%%%%%%%%%%%%%%%%%%%%%%%%%%%%%%%%%%%%%%%%%%%%%%%%%%%%%%%%%%%%%%%%%%%%%%%%%%

\end{document}
